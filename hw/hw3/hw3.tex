\documentclass{../homework}
\usepackage{ dsfont }

\name{Timothy Devon Morris}
\course{Me En 537}
\term{Fall 2018}
\hwnum{3}

\begin{document}

\begin{problem}[Corke Chap 7]
\end{problem}

\begin{solution}
  \begin{parts}
    \part{7-2}
    I ran \texttt{mdl\_puma560} and then played around with \texttt{p560.teach()}.
    \part{7-6}
    See \texttt{prob76.m}
  \end{parts}
\end{solution}

\begin{problem}
\end{problem}

\begin{solution}
  \begin{parts}
    \part
    This robot has DH parameters of
       \begin{center}
       \begin{tabular}{|c|c|c|c|r|}
        \hline
        Link & $\theta_i$ & $d_i$ & $a_i$ & $\alpha_i$ \\
        \hline
        1 & $\theta_1$ & $d_1$ & 0 & $\pi/2$ \\
        2 & $\theta_2$ & 0 & $a_2$ & 0 \\
        3 & $\theta_3$ & 0 & $a_3$ & 0 \\
        \hline
       \end{tabular}
       \end{center}

       This gives us the following geometric jacobian (using the toolbox)
       \[
         J(q) = 
         \begin{bmatrix}
           -s_{\theta_1}(a_3c_{\theta_2 + \theta_3} + a_2c_{\theta_3}) & -c_{\theta_1}(a_3s_{\theta_2 + \theta_3} + a_2s_{\theta_2}) & -a_3s_{\theta_2 + \theta_3}c_{\theta_1} \\
           c_{\theta_1}(a_3c_{\theta_2 + \theta_3} + a_2c_{\theta_2}) & -s_{\theta_1}(a_3s_{\theta_2 + \theta_3} + a_2s_{\theta_2}) & -a_3s_{\theta_2 + \theta_3}s_{\theta_1} \\
           0 & a_3c_{\theta_2 + \theta_3} + a_2c_{\theta_2} & a_3c_{\theta_3 + \theta_3} \\
           0 & s_{\theta_1} & s_{\theta_1} \\
           0 & -c_{\theta_1} & -c_{\theta_1} \\
           1 & 0 & 0
         \end{bmatrix}
       \]
       See \texttt{prob2a.m} for some calculations of the geometric jacobians. One example that we have is $q = [0\ \pi/2\ 0]^T$, in this case we have
       \[
         J\left(
         \begin{bmatrix}
           0 \\ 
           \pi/2 \\
           0
         \end{bmatrix}
         \right)
         =
         \begin{bmatrix}
           0 &  -0.6 & -0.3 \\
           0 & 0 & 0 \\
           0 & 0 & 0 \\
           0 & 0 & 0 \\
           0 & -1 & -1 \\
           1 & 0 & 0
         \end{bmatrix}
       \]
       In this case, the robot arm is pointing straight up. Intuitively, the arm can only have x linear velocity and can only have angular velocity in y and z.
       \part
       We have that the DH parameters up to the point $o_c$ are
       \begin{center}
       \begin{tabular}{|c|c|c|c|r|}
        \hline
        Link & $\theta_i$ & $d_i$ & $a_i$ & $\alpha_i$ \\
        \hline
        1 & $\theta_1$ & 0 & $a_1$ & 0 \\
        2 & $\theta_2$ & 0 & $a_c$ & 0 \\
        \hline
       \end{tabular}
       \end{center}

       Which gives us the matrices
       \[
         A_1 = \begin{bmatrix}
           c_{\theta_1} & -s_{\theta_1} & 0 & a_1c_{\theta_1} \\
           s_{\theta_1} & c_{\theta_1} & 0 & a_1cs_{\theta_1} \\
           0 & 0 & 1 & 0 \\
           0 & 0 & 0 & 1
         \end{bmatrix}
       \]
       \[
         A_2 = \begin{bmatrix}
           c_{\theta_2} & -s_{\theta_2} & 0 & a_cc_{\theta_1} \\
           s_{\theta_2} & c_{\theta_2} & 0 & a_cs_{\theta_1} \\
           0 & 0 & 1 & 0 \\
           0 & 0 & 0 & 1
         \end{bmatrix}
       \]
       And thus we have
       \[
         ^0T_c = 
         \begin{bmatrix}
           c_{\theta_1 + \theta_2} & -s_{\theta_1 + \theta_2} & 0 & a_c c_{\theta_1+\theta_2} + a_1c_{\theta_1} \\
           s_{\theta_1 + \theta_2} & c_{\theta_1 + \theta_2} & 0 & a_cs_{\theta_1+\theta_2} + a_1s_{\theta_1} \\
           0 & 0 & 1 & 0 \\
           0 & 0 & 0 & 1
         \end{bmatrix}
       \]
       Specifically, we have that the forward kinematics of the point $o_c$ are
       \[
         o_c^0(\theta_1,\theta_2) = 
         \begin{bmatrix}
           a_c c_{\theta_1 + \theta_2} + a_1c_{\theta_1} \\
           a_c s_{\theta_1 + \theta_2} + a_1s_{\theta_1} \\
           0
         \end{bmatrix}
       \]
       Since we get no contribution from the third joint, our (zero frame) jacobian becomes
       \[
         \begin{aligned}
           J(q) &= 
         \begin{bmatrix}
           z_0^0 \times (o_c^0) & z_1^0 \times (o_c^0 - o_1^0) & 0 \\
           z_0^0 & z_1^0 & 0 
         \end{bmatrix} \\
         &= 
         \begin{bmatrix}
           a_1c_{\theta_1 + \theta_2}s_{\theta_2} - s_{\theta_1 + \theta_2}(a_c + a_1c_{\theta_2}) & -a_cs_{\theta_1 + \theta_2} & 0\\
           c_{\theta_1 + \theta_2}(a_c + a_1c_{\theta_2}) + a_1s_{\theta_1 + \theta_2}s_{\theta_2} & a_c c_{\theta_1 + \theta_2} & 0\\
           0 & 0 & 0 \\
           0 & 0 & 0 \\
           0 & 0 & 0 \\
           1 & 1 & 0
         \end{bmatrix}
         \end{aligned}
       \]
     \part
     This robot has the following DH parameters
     \begin{center}
     \begin{tabular}{|c|c|c|c|r|}
      \hline
      Link & $\theta_i$ & $d_i$ & $a_i$ & $\alpha_i$ \\
      \hline
      1 & $\theta_1$ & 0 & $a_1$ & $-\pi/2$ \\
      2 & $\theta_2$ & 0 & $a_2$ & 0 \\
      \hline
     \end{tabular}
     \end{center}

     Which gives us the forward kinematics of
     \[
       A_1 = 
       \begin{bmatrix}
         c_{\theta_1} & 0 & -s_{\theta_1} & a_1c_{\theta_1} \\
         s_{\theta_1} & 0 & c_{\theta_1} & a_1s_{\theta_1} \\
         0 & -1 & 0 & 0 \\
         0 & 0 & 0 & 1
       \end{bmatrix}
     \]
     \[
       A_2 =
       \begin{bmatrix}
         c_{\theta_2} & -s_{\theta_2} & 0 & a_2c_{\theta_2} \\
         s_{\theta_2} & c_{\theta_2} & 0 & a_2s_{\theta_2} \\
         0 & 0 & 1 & 0 \\
         0 & 0 & 0 & 1
       \end{bmatrix}
     \]
     And our transform is
     \[
       ^0T_2 = 
       \begin{bmatrix}
         c_{\theta_1}c_{\theta_2} & -s_{\theta_2}c_{\theta_1} & - s_{\theta_1} & a_2c_{\theta_1}c_{\theta_2} + a_1c_{\theta_1} \\
         s_{\theta_1}c_{\theta_2} & -s_{\theta_1}s_{\theta_2} & c_{\theta_1} & a_2s_{\theta_1}c_{\theta_2} + a_1s_{\theta_1} \\
         -s_{\theta_2} & -c_{\theta_2} & 0 & -a_2s_{\theta_2} \\
         0 & 0 & 0 & 1
       \end{bmatrix}
     \]
     In this case, we know that the jacobian of the end effector in the end effector frame must be
     \[
      ^2J(q) = 
      \begin{bmatrix}
        z_0^2 \times (o^2_2 - o^2_0) & z_1^2 \times (o^2_2 - o^2_1) \\
        z_0^2 & z_1^2
      \end{bmatrix}
     \]
     Putting everything in the right frame, we have the following
     \[
       z_0^2 =\ ^2R_0z_0^0 =
       \begin{bmatrix}
         -s_{\theta_2} \\
         -c_{\theta_2} \\
         0
       \end{bmatrix}
     \]
     \[
       o_2^2 = \begin{bmatrix}
         0 \\
         0 \\
         0
       \end{bmatrix}
     \]
     \[
       \begin{bmatrix}
         o_0^2 \\
         1
       \end{bmatrix}
       =\ ^2T_0
       \begin{bmatrix}
         o_0 \\
         1
       \end{bmatrix}
       =
       \begin{bmatrix}
         -a_2 - a_1c_{\theta_2} \\
         a_1s_{\theta_2} \\
         0 \\
         1
       \end{bmatrix}
     \]
     \[
       o_1^2 = 
       \begin{bmatrix}
         a_2 \\
         0 \\
         0
       \end{bmatrix}
     \]
     \[
       z_1^2 =  
       \begin{bmatrix}
       0 \\
       0 \\
       1
       \end{bmatrix}
     \]
     Which gives us the geometric jacobian as 
     \[
      ^2J(q) = 
      \begin{bmatrix}
        0 & 0 \\
        0 & a_2 \\
        a_1 + a_2c_{\theta_2} & 0 \\
        -s_{\theta_2} & 0 \\
        -c_{\theta_2} & 0 \\
        0 & 1
      \end{bmatrix}
     \]
     Now we will write the Jacobian in the base frame and transform it into the end effector frame.
     \[
      ^0J(q) = 
      \begin{bmatrix}
        z_0^0 \times (o^0_2 - o^0_0) & z_1^0 \times (o^0_2 - o^0_1) \\
        z_0^0 & z_1^0
      \end{bmatrix}
     \]
     Substituting in the correct values from our forward kinematics, we have that
     \[
       ^0J(q) = 
       \begin{bmatrix}
         -a_1s_{\theta_1} - a_2s_{\theta_1}c_{\theta_2} & -a_2c_{\theta_1}s_{\theta_2} \\
         a_1c_{\theta_1} - a_2c_{\theta_1}c_{\theta_2} & -a_2s_{\theta_1}s_{\theta_2} \\
         0 & -a_2c_{\theta_2} \\
         0 & -s_{\theta_1} \\
         0 & c_{\theta_1} \\
         1 & 0 
       \end{bmatrix}
     \]
     And we have
     \[
       ^2J(q) =
       \begin{bmatrix}
         ^2R_0 & 0 \\
         0 & ^2R_0 
       \end{bmatrix}\ ^0J(q)
     \]
     Where
     \[^2R_0 = 
       \begin{bmatrix}
         c_{\theta_1}c_{\theta_2} & -s_{\theta_2}c_{\theta_1} & - s_{\theta_1}\\
         s_{\theta_1}c_{\theta_2} & -s_{\theta_1}s_{\theta_2} & c_{\theta_1} \\
         -s_{\theta_2} & -c_{\theta_2} & 0\\
       \end{bmatrix}^T
     \]
     This gives us
     \[
       \begin{bmatrix}
         ^2R_0 & 0 \\
         0 & ^2R_0 
       \end{bmatrix}\
       ^0J(q) = 
      \begin{bmatrix}
        0 & 0 \\
        0 & a_2 \\
        a_1 + a_2c_{\theta_2} & 0 \\
        -s_{\theta_2} & 0 \\
        -c_{\theta_2} & 0 \\
        0 & 1
      \end{bmatrix}
     \]
     So the answers are consistent!
     \part
     See \texttt{geo\_jacobian.m}. I checked this function in various configurations on the stanford robot.
     \part
     We have the relation that $\tau = J^T(q)F$. Plugging this in for various configurations gives
     \begin{parts}[r]
      \part 
      \[
        J^T \left( \begin{bmatrix}
          0 \\
          \pi/4
        \end{bmatrix} \right)
        \begin{bmatrix}
          -1\\ 
          0 \\
          0 \\
          0 \\
          0 \\
          0
        \end{bmatrix}
        = 
        \begin{bmatrix}
          0 \\
          a_2/\sqrt{2}
        \end{bmatrix}
      \]
      \part
      \[
        J^T \left( \begin{bmatrix}
          0 \\
          \pi/2
        \end{bmatrix} \right)
        \begin{bmatrix}
          -1\\ 
          0 \\
          0 \\
          0 \\
          0 \\
          0
        \end{bmatrix} = 
        \begin{bmatrix}
          0 \\
          a_2
        \end{bmatrix}
      \]
      \part
      \[
        J^T \left( \begin{bmatrix}
          \pi/4 \\
          \pi/4
        \end{bmatrix} \right)
        \begin{bmatrix}
          -1\\ 
          -1 \\
          0 \\
          0 \\
          0 \\
          0
        \end{bmatrix} = 
        \begin{bmatrix}
          0 \\
          a_2
        \end{bmatrix}
      \]
      \part
      \[
        J^T \left( \begin{bmatrix}
          0 \\
          0 
        \end{bmatrix} \right)
        \begin{bmatrix}
          0\\ 
          0 \\
          1 \\
          0 \\
          0 \\
          0
        \end{bmatrix} = 
        \begin{bmatrix}
          0 \\
          -a_2
        \end{bmatrix}
      \]
      \part
      \[
        J^T \left( \begin{bmatrix}
          0 \\
          0 
        \end{bmatrix} \right)
        \begin{bmatrix}
          1\\ 
          0 \\
          0 \\
          0 \\
          0 \\
          0
        \end{bmatrix} = 
        \begin{bmatrix}
          0 \\
          0
        \end{bmatrix}
      \]
      If the force is a reaction force, you will need to change the sign on the force vector, thus negating the joint torques.

     \end{parts}
     
  \end{parts}
\end{solution}

\end{document}
