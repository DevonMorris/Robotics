\documentclass{../homework}
\usepackage{ dsfont }

\name{Timothy Devon Morris}
\course{Me En 537}
\term{Fall 2018}
\hwnum{1}

\begin{document}

\begin{problem}[Corke Chap 2]
\end{problem}

\begin{solution}
  \begin{parts}
    \part
    See \texttt{prob2.m}
    \part
    See \texttt{prob4.m}
    \part
    See \texttt{prob5.m}
    \part
    See \texttt{prob10.m}
    \part
    No, consider the following homogeneous transformation
    \[ T =
      \begin{bmatrix}
        1 & 0 & 0 & 1 \\
        0 & 1 & 0 & 1 \\
        0 & 0 & 1 & 1 \\
        0 & 0 & 0 & 1 \\
      \end{bmatrix}
    \]
    Note that $TT^T$ is not the identity
    \[
      TT^T = 
      \begin{bmatrix}
        2 & 1 & 1 & 1\\
        1 & 2 & 1 & 1\\
        1 & 1 & 2 & 1\\
        1 & 1 & 1 & 1
      \end{bmatrix}
    \]
    In fact it is not even a homogeneous transformation.
  \part
  Note since we are in a right handed coordinate frame, and we have two vectors, the other must be completely determined. Thus we have
  \[
    ^0R_1 = 
    \begin{bmatrix}
      1 & 0 & 0 \\
      0 & 0 & 1 \\
      0 & -1 & 0
    \end{bmatrix}
  \]
  And converting to quaternion gives us
  \[
    ^0q_1 = .707 \langle -.707, 0, 0\rangle
  \]

  \end{parts}
\end{solution}

\begin{problem}
\end{problem}

\begin{solution}
 \begin{parts}
   \part
   \[^0R_3 = 
     \begin{bmatrix}
     1 &     0    &    0      \\
     0 & \cos(\phi) & -\sin(\phi) \\
     0 & \sin(\phi) & \cos(\phi)
     \end{bmatrix}
     \begin{bmatrix}
     \cos(\theta) & -\sin(\theta) & 0 \\
     \sin(\theta) & \cos(\theta)  & 0 \\
     0     &    0      & 1
     \end{bmatrix}
     \begin{bmatrix}
     \cos(\psi)  & 0 & \sin(\psi) \\
     0 			  & 1 &    0     \\
     -\sin(\psi) & 0 & \cos(\psi)
     \end{bmatrix}
   \]
 \part
   \[^0R_3 = \begin{bmatrix}
   1 &     0    &    0      \\
   0 & \cos(\phi) & -\sin(\phi) \\
   0 & \sin(\phi) & \cos(\phi)
   \end{bmatrix}
   \begin{bmatrix}
   \cos(\theta) & -\sin(\theta) & 0 \\
   \sin(\theta) & \cos(\theta)  & 0 \\
   0     &    0      & 1
   \end{bmatrix}
   \begin{bmatrix}
   \cos(\psi)  & 0 & \sin(\psi) \\
   0 			  & 1 &    0     \\
   -\sin(\psi) & 0 & \cos(\psi)
   \end{bmatrix}
   \]
 \end{parts} 
\end{solution}

\begin{problem}
\end{problem}

\begin{solution}
  \[
  ^0R_1 =
  \begin{bmatrix}
  \cos(\pi/2)  & 0 & \sin(\pi/2) \\
  0 			  & 1 &    0     \\
  -\sin(\pi/2) & 0 & \cos(\pi/2)
  \end{bmatrix}
  \begin{bmatrix}
  1 &     0    &    0      \\
  0 & \cos(\pi/2) & -\sin(\pi/2) \\
  0 & \sin(\pi/2) & \cos(\pi/2)
  \end{bmatrix}
  =
  \begin{bmatrix}
    0 & 1 & 0 \\
    0 & 0 & -1 \\
    -1 & 0 & 0
  \end{bmatrix}
  \]
If you sketched this, it would look very similar to the original frame, but with the new x being the old y, the new y being the old z and the new z being the old x.
\end{solution}

\begin{problem}
\end{problem}

\begin{solution}
 Note we have that 
 \[^2R_3 =\ ^2R_1\ ^1R_3 = (^1R_2)^T\ ^1R_3 = 
 \begin{bmatrix}
  0 & 0 & -1 \\
  \frac{\sqrt{3}}{2} & \frac{1}{2} & 0\\
  \frac{1}{2} & -\frac{\sqrt{3}}{2} & 0
 \end{bmatrix}
 \]
\end{solution}

\begin{problem}
\end{problem}

\begin{solution}
 Using Matlab symbolic variables, I got
 \[ 
   \begin{aligned}
     R_{x,\theta}&R_{y,\phi}R_{z,\pi}R_{y,-\phi}R_{x,-\theta} = \\
                                                            &\begin{bmatrix}
     2\sin(\phi)^2 -1 & -2\cos(\phi)\sin(\phi)\sin(\theta) & 2\cos(\phi)\cos(\theta)\sin(\phi) \\
     -2\cos(\phi)\sin(\phi)\sin(\theta) & 2\sin(\theta)^2 - 2\sin(\phi)^2\sin(\theta)^2 -1 & -2\cos(\phi)^2\cos(\theta)\sin(\theta) \\
     2\cos(\phi)\cos(\theta)\sin(\phi) & -2\cos(\phi)\cos(\theta)\sin(\theta) & 2\cos(\phi)^2\cos(\theta)^2 -1
   \end{bmatrix}
 \end{aligned}
 \]
\end{solution}

\begin{problem}
\end{problem}

\begin{solution}
  Using the matlab toolbox, the axis-angle representation is
  \[
    \theta = 1.71777, \quad k =
    \begin{bmatrix}
      0.357407 \\
      0.862856 \\
      0.357407
    \end{bmatrix}
  \]
  The quaternion representation is 
  \[
    q = 0.65328 \langle 0.2706, 0.65328, 0.2706 \rangle
  \]
\end{solution}

\begin{problem}
\end{problem}

\begin{solution}
  Using the ZYZ euler angles,
  \[ ^0R_3 = R_{z,\pi/2}IR_{z,\pi/4} = R_{z, 3 \pi/4} =
  \begin{bmatrix}
  \cos(3\pi/4) & -\sin(3\pi/4) & 0 \\
  \sin(3\pi/4) & \cos(3\pi/4)  & 0 \\
  0     &    0      & 1
  \end{bmatrix}
  \]
  Which implies that the new x direction is
  \[^0x_1 =
  \begin{bmatrix}
    -\frac{\sqrt{2}}{2} \\
    \frac{\sqrt{2}}{2} \\
    0
  \end{bmatrix}\]
\end{solution}

\end{document}
